\documentclass[11pt]{article}
\usepackage[letterpaper,margin=1in,centering]{geometry}
\usepackage{amssymb}  
\usepackage{amsmath}
\usepackage{amsfonts}
\usepackage{mathtools}
\usepackage{microtype}
\usepackage{color}
\usepackage{hyperref}
\usepackage[]{cleveref} % get fancy referencing


%%%%%%%%%%%%%%%%%%%%%%%%%
% CUSTOM MACROS
%%%%%%%%%%%%%%%%%%%%%%%%%
\newcommand{\linearize}[3]{\ensuremath{\left. \frac{\partial #1}{\partial #2} \right|_{#3} \delta #2}}
\newcommand{\norm}[1]{\ensuremath{\left\| #1 \right\|}}
\newcommand{\Real}[1]{\ensuremath{\Re \left\{ #1 \right\}}}
\newcommand{\bracket}[1]{\ensuremath{\left[ #1 \right]}}
\newcommand{\braces}[1]{\ensuremath{\left\{ #1 \right\}}}
\newcommand{\parenth}[1]{\ensuremath{\left( #1 \right)}}
\newcommand{\pair}[1]{\ensuremath{\langle #1 \rangle}}
\newcommand{\met}[1]{\ensuremath{\langle\langle #1 \rangle\rangle}}
\newcommand{\refeqn}[1]{(\ref{eqn:#1})}
\newcommand{\reffig}[1]{Fig. \ref{fig:#1}}
\newcommand{\tr}[1]{\mathrm{tr}\ensuremath{\negthickspace\bracket{#1}}}
\newcommand{\trs}[1]{\mathrm{tr}\ensuremath{[#1]}}
\newcommand{\deriv}[2]{\ensuremath{\frac{\partial #1}{\partial #2}}}
\newcommand{\diff}[2]{\ensuremath{\frac{d #1}{d #2}}}
\newcommand{\dirDiff}[2]{\ensuremath{\mathbf{D}_{#2} #1 \cdot \delta #2}} % directional derivative
\newcommand{\SO}{\ensuremath{\mathsf{SO(3)}}}
\newcommand{\T}{\ensuremath{\mathsf{T}}}
\renewcommand{\L}{\ensuremath{\mathsf{L}}}
\newcommand{\so}{\ensuremath{\mathfrak{so}(3)}}
\newcommand{\SE}{\ensuremath{\mathsf{SE(3)}}}
\newcommand{\se}{\ensuremath{\mathfrak{se}(3)}}
\newcommand{\R}{\ensuremath{\mathbb{R}}}
\newcommand{\aSE}[2]{\ensuremath{\begin{bmatrix}#1&#2\\0&1\end{bmatrix}}}
\newcommand{\ase}[2]{\ensuremath{\begin{bmatrix}#1&#2\\0&0\end{bmatrix}}}
\newcommand{\D}{\ensuremath{\mathbf{D}}}
\newcommand{\Sph}{\ensuremath{\mathsf{S}}}
\renewcommand{\S}{\Sph}
\newcommand{\J}{\ensuremath{\mathbf{J}}}
\newcommand{\Ad}{\ensuremath{\mathrm{Ad}}}
\newcommand{\intp}{\ensuremath{\mathbf{i}}}
\newcommand{\extd}{\ensuremath{\mathbf{d}}}
\newcommand{\hor}{\ensuremath{\mathrm{hor}}}
\newcommand{\ver}{\ensuremath{\mathrm{ver}}}
\newcommand{\dyn}{\ensuremath{\mathrm{dyn}}}
\newcommand{\geo}{\ensuremath{\mathrm{geo}}}
\newcommand{\Q}{\ensuremath{\mathsf{Q}}}
\newcommand{\G}{\ensuremath{\mathsf{G}}}
\newcommand{\g}{\ensuremath{\mathfrak{g}}}
\newcommand{\Hess}{\ensuremath{\mathrm{Hess}}}
\newcommand{\refprop}[1]{Proposition \ref{prop:#1}}
\newcommand{\mypaper}{}

\newcommand{\RNum}[1]{\uppercase\expandafter{\romannumeral #1\relax}}
\newcommand{\RI}{\text{\RNum{1}}}
\newcommand{\RII}{\text{\RNum{2}}}
\newcommand{\RIII}{\text{\RNum{3}}}

\newenvironment{correction}{\begin{list}{}{\setlength{\leftmargin}{1cm}\setlength{\rightmargin}{1cm}}\vspace{\parsep}\item[]``}{''\end{list}}

\begin{document}

%\pagestyle{empty}

\section*{Response to the Reviewers' Comments for JCAS-D-16-00607}

I would like to thank the reviewers for their thoughtful comments, which are aimed
towards improving the quality of the paper and the clarity of the results. In accordance with the comments and suggestions, the paper has been revised, and the answers to all comments are addressed as follows.

(In the revised manuscript, the citation numbers for equations, assumptions, propositions, and references are changed. This answer is written according to the new item numbers.)

\subsection*{Reviewer 2}

\textit{A new geometric adaptive control scheme for the stabilization of the attitude dynamics of a rigid body with state inequality constraints is studied in this paper and ensures that the closed-loop system is asymptotically stable in the sense of Lyapunov stability. The main contribution and result are well presented and organized. Therefore, it is sufficient to be accepted in this journal provided that the following minor problems are considered in the revised version.
}

% \setlength{\leftmargini}{0pt}
\begin{enumerate}

\item \textit{ In general, the external disturbance is a time-varying signal even though it is bounded. Thus, authors should consider the time-varying disturbance case.}

It is true that in general, the external disturbances on a system will be time varying.
However, the form of uncertainty shown in \( (1) \) is commonly used in the adaptive control literature~\cite{lee2013b,ioannou2012}.
As a result, we use this same form of uncertainty, namely that the uncertain disturbance is additive in nature and enters through the input channel.
In addition, a wide variety of disturbances is accurately represented by this formulation in the aerospace engineering field and several examples are given in the article.
Furthermore, it has been shown in~\cite{ioannou2012} that this form of adaptive control is able to handle time-varying uncertainties in which the uncertain term is varying sufficiently slowly as compared to the controlled dynamics. 

How slowly do the uncertainties have to behave for the simulation?

\item \textit{If the inertia matrix is unknown, how to modify the suggested control system guaranteeing the asymptotic stability?}
Even with an uncertainty in the inertia matrix, one could redefine the system and treat this as an unknonw external disturbance instead.

Show the algebra that allows for either a multiplicative or additive uncertainty in the inertia matrix as equivalent to an external torque

\item \textit{The state inequality constraint is not violated based on adding the repulsive term in the attitude error function. Is this result still valid if this constraint becomes time-dependent?}

Run another simulation showing that for a restricted class of time-varying error functions, the system is stable 

\end{enumerate}

\subsection*{Reviewer 3}

\textit{This paper considers the attitude control with inequality constraints. Attitude dynamics with matched uncertainty which is parameterized by a unknown constant matrix, and hard constants expressed by rotation matrix are considered. Although the problem is very interesting, the proofs for major results on the case with plant uncertainty are far from complete. Some comments are listed below.}

\begin{enumerate}
\item \textit{Please give a precise definition of variation used in the paper, e.g., eq. (17) and (18)}
\item \textit{This is serious. In section 2, the upper bounds listed in Proposition 4 are given. Well, these should be the properties that your controller guarantees not come from the set D you've defined. Since these properties are used to derive the bound H which is used to derive the result in Proposition 5, the stability proof under the proposed controller is far from complete.}
\item \textit{Typo in eq. (40).}
\item \textit{Abuse of notation 'hat' and 'vee' in equations (6)-(8)}
\end{enumerate}

\subsection*{Reviewer 4}
\textit{This paper presents a new geometric adaptive control system with state inequality constraints for the attitude dynamics of a rigid body. The controlled attitude trajectory avoids undesired regions defined by the inequality constraint, and an adaptive update law that enables attitude stabilization in the presence of unknown disturbances is developed. However, the paper can be improved and some revisions are required.}

\begin{enumerate}
\item \textit{Probably, the weak point is that the underlying experiments are not solid. The author claims that the attitude dynamics and the proposed control systems are developed on the special orthogonal group such that singularities and ambiguities of other attitude parameterizations, such as Euler angles and quaternions are completely avoided.  But the numerical simulations and experimental results don't demonstrate the effectiveness of the proposed control system. How is the singularity avoided in the Euler angles? Please give the detail.}
\item \textit{The units of y axis of Figures 1 - 3 and 5 should be mentioned in the manuscript.}
\item \textit{Please check and correct Page 2 Line 60.}
\item \textit{The units of the parameters are missing. These should be added in the manuscript.}
\item \textit{What are the design parameters of the proposed method and how the design parameters are selected? These should be included in the manuiscript.}
\item \textit{Specification of the sensors in the experiments should be included in the manuscript.}
\end{enumerate}

\subsection*{Reviewer 5}
\textit{This paper studies adaptive control design for the attitude dynamics of a rigid body system with state inequality constraints. The designed controller is proved to be asymptotically stabilized and may enable attitude stabilization in the presence of unknown disturbances. The attitude trajectory by the controller may avoid undesired regions. The effectiveness of the proposed control system is demonstrated through numerical simulations and experimental results. This paper is well written and the research contents are interesting. The main contributions of this paper are the new geometric adaptive control system proposed by the authors and its experimental verification. This paper is well organized and almost need not to make any changes. This reviewer suggests that it can be accepted for publication in International Journal of Control, Automation and Systems.}

\begin{enumerate}
\item \textit{The only question is that all the figures in this paper are too small and are not legible. The meanings of different lines in one figure should be explained in the title content.}
\end{enumerate}

\bibliography{library}
\bibliographystyle{ieeetr}
\end{document}

